\documentclass[12pt,a4paper]{article}

\usepackage{tikz}
\usetikzlibrary{patterns}
\usetikzlibrary{graphs, graphs.standard, quotes}
\usetikzlibrary{positioning, arrows, automata}
\usepackage{tkz-berge}
\usepackage{graphicx}
\usepackage{caption}
\usepackage{subcaption}
\usepackage{tabularx}
\usepackage{algorithmic}
\usepackage{algorithm2e}
\usepackage[utf8]{inputenc}
\usepackage{wrapfig}
\usepackage{float}
\usepackage{etoolbox}
\usepackage{xcolor}[dvipsnames]
\usepackage{keyval}
\usepackage[shortlabels]{enumitem}
\usepackage{multicol}
\setlength{\columnsep}{5pt}
\raggedcolumns 

\def\Side{\ChessSide}
\newcommand\ChessBoxA{%
  {\fboxsep=0pt\fbox{\color{\ChessColori}\rule{\Side}{\Side}}}}
\newcommand\ChessBoxB{%
  {\fboxsep=0pt\fbox{\color{\ChessColorii}\rule{\Side}{\Side}}}}

\makeatletter
\newcommand\Row[1]{%
  \par\nobreak\nointerlineskip\vskip-\fboxrule%
  \@tfor\@tempa:=#1 \do {\csname ChessBox\@tempa\endcsname\kern-\fboxrule}}
\define@key{chessB}{side}{\def\ChessSide{#1}}
\define@key{chessB}{colori}{\def\ChessColori{#1}}
\define@key{chessB}{colorii}{\def\ChessColorii{#1}}
\setkeys{chessB}{
  side=1.5em,
  colori=black!70,
  colorii=white}
\makeatother

\newcommand\Conventional[1][]{%
\begin{Chessboard}[#1]
\Row{B,A,B,A,B,A,B,A}
\Row{A,B,A,B,A,B,A.B}
\Row{B,A,B,A,B,A,B,A}
\Row{A,B,A,B,A,B,A.B}
\Row{B,A,B,A,B,A,B,A}
\Row{A,B,A,B,A,B,A.B}
\Row{B,A,B,A,B,A,B,A}
\Row{A,B,A,B,A,B,A.B}
\end{Chessboard}%
}

\newenvironment{Chessboard}[1][]
  {\setkeys{chessB}{#1}%
  \par\medskip\setlength\parindent{0pt}}
  {\par\medskip}

\renewcommand{\familydefault}{\rmdefault}
\renewcommand{\refname}{Referencias}

\BeforeBeginEnvironment{figure}{\vskip-2ex}
\AfterEndEnvironment{figure}{\vskip-2ex}

\begin{document}

\begin{titlepage}
	\centering
	\includegraphics[width=0.30\textwidth]{images/Teclogocompleto.jpg}\par\vspace{1cm}
	{\scshape\large \textbf{Instituto Tecnológico de Costa Rica }\par}
	\vspace{1cm}
	{\scshape\Large MC 6102 Análisis y diseño del Algoritmos\par}
	\vspace{1.5cm}
	{\Large\bfseries Tarea \#4\par}
	\vspace{2cm}
	{\Large\itshape Ricardo Alfaro Villalobos\par}
	\vfill
	Profesor:\par
	Jose Araya Monge\textsc{}

	\vfill

% Bottom of the page
	{\large 18 de octubre del 2019\par}
\end{titlepage}

\begin{section}{Tablero y guijarros.} \noindent 
\begin{enumerate}[a., nolistsep, font=\bfseries]
\setlength\multicolsep{0pt}
\item \textbf{Numero de patrones.} Cada columna puede tener 8 patrones legales, 1 vacío(sin ningún guijarro en ninguna celda), 4 con un solo guijarro en cualquiera de las 4 celdas y 3 con dos guijarros (el las celdas 1 y 4, 1 y 3 o 2 y 4). La siguiente imagen describe de manera gráfica estos patrones.  

\begin{center}
\begin{multicols}{8}
\begin{Chessboard}
\textbf{1}
\vspace{1em}
\Row{B}
\Row{B}
\Row{B}
\Row{B}
\end{Chessboard}
\columnbreak
\begin{Chessboard}
\textbf{2}
\vspace{1em}
\Row{A}
\Row{B}
\Row{B}
\Row{B}
\end{Chessboard}
\columnbreak
\begin{Chessboard}
\textbf{3}
\vspace{1em}
\Row{B}
\Row{A}
\Row{B}
\Row{B}
\end{Chessboard}
\columnbreak
\begin{Chessboard}
\textbf{4}
\vspace{1em}
\Row{B}
\Row{B}
\Row{A}
\Row{B}
\end{Chessboard}
\columnbreak
\begin{Chessboard}
\textbf{5}
\vspace{1em}
\Row{B}
\Row{B}
\Row{B}
\Row{A}
\end{Chessboard}
\columnbreak
\begin{Chessboard}
\textbf{6}
\vspace{1em}
\Row{A}
\Row{B}
\Row{B}
\Row{A}
\end{Chessboard}
\columnbreak
\begin{Chessboard}
\textbf{7}
\vspace{1em}
\Row{A}
\Row{B}
\Row{A}
\Row{B}
\end{Chessboard}
\columnbreak
\begin{Chessboard}
\textbf{8}
\vspace{1em}
\Row{B}
\Row{A}
\Row{B}
\Row{A}
\end{Chessboard}
\end{multicols}
\end{center}

Cada patrón tiene un numero determinado de patrones compatibles, por lo que al colocar un nuevo patrón en la posición $k+1$, en subproblemas de tamaño $k$, es posible determinar si se se ha logrado la solución optima a ese punto o si existe otra combinación valida de patrones que de un máximo en comparación con el actual.\\

\item \textbf{Algoritmo.} Crear una matriz de compatibilidad entre patrones en un paso previo a calcular la mejor configuración posible de guijarros en el tablero . Para calcular la configuración optima de guijarros: crear un arreglo de $n$ elementos para cada patrón valido (8 en total). En un ciclo desde $1 ... n$, ir almacenando en cada arreglo la configuración valida de columnas que permite obtener el valor máximo en cada caso y en una variable ir actualizando la suma máxima. Esto se puede hacer almacenando la suma del valor obtenido hasta el momento mas el valor de las celdas cubiertas por $x$, si al agregar la siguiente columna de tipo $x'$ en la posición $n+1$ se puede obtener un máximo mas alto mediante otra configuración de tipo $x$ entonces se puede reconstruir $1...x$ siempre y cuando $x$ siga siendo compatible con $x'$. Este algoritmo es de orden $O(n)$ por que el llenado de el valor constante de arreglos se hace en un solo ciclo.  
\end{enumerate}
\end{section}
\newpage

\begin{section}{Fourier Fast Transform, FFT.} \noindent 
\begin{enumerate}[a., font=\bfseries]
\item \textbf{Problema.} Se tienen los siguientes polinomios:
\begin{center}
$p(x)=a_0+a_1x+...+a_{n-1}x^{n-1}$\\
\vspace{1em}
$q(x)=b_0+b_1x+...+b_{x-1}x^{n-1}$
\end{center}
Cuyo producto esta definido por:
\begin{center}
$p(x)\cdot q(x) = c_0+c_1x+...+c_{2n-2}x^{2n-2}$
\end{center}
Para calcular el producto de una multiplicación de polinomios cada $a_i$ es multiplicado con cada $b_j$, para $0\leq i,j \leq n-1$, esto representa hasta $n^2$ multiplicaciones, dado que algunos coeficientes pueden ser cero. Obtener cada $c_i$ implica algunas menos adiciones que multiplicaciones, entonces se tiene que hay hasta $n^2 - 2n+1$ adiciones. En conclusión el numero de operaciones aritméticas es $O(n^2)$. \cite{jia2014polynomial}
\item \textbf{Algoritmo.}
\item \textbf{$(1-2x-x^2+x^3)*(2+x-4x^2+x^3)=2-3x-8x^2+10x^3+3x^4-5x^5+x^6$}
\end{enumerate}
\end{section}

\newpage
\nocite{*}
\bibliographystyle{ieeetr}
\bibliography{bibliography/ref}
\end{document}

